\documentclass{bmstu}
\usepackage{multirow}
%\usepackage{karnaugh-map}
\usepackage{tikz}
\usepackage{float}

\usetikzlibrary{karnaugh}

\begin{document}

	\makecourseworktitle
	{Радиоэлектроника и лазерная техника (РЛ)} % Название факультета
	{Технология приборостроения (РЛ6)} % Название кафедры
	{Проектирование конструкции микрополоскового детектора} % Тема работы
	{Филимонов~С.~В./РЛ6-61} % Номер группы/ФИО студента (если авторов несколько, их необходимо разделить запятой)
	{Федоркова Н.В.} % ФИО научного руководителя
	{}
	
	\tableofcontents
	
	\chapter{Принцип работы устройства.}
	
	Амплитудный детектор – это устройство, предназначенное для получения на
	выходе напряжения, изменяющегося в соответствии с законом модуляции
	амплитуды входного гармонического сигнала. Процесс детектирования
	амплитудно-модулированных (АМ) сигналов вида
	
	\begin{center}
		 $U_c(t) = u_a(t) \cdot cos(\omega _c \cdot t), \text{где } u_a(t)=U_c[1 + max(t)],$
	\end{center}
	
	где $m_a$ < 1 - коэффициент глубины модуляции, $U_c$ амплитуда несущего 
	колебания с частотой $\omega _c$, заключается в воспроизведении модулирующего 
	сообщения $x(t)$ с наименьшими искажениями. Спектр сообщения $x(t)$ 
	сосредоточен в области низких частот (частот модуляции), а спектр сигнала 
	$U_c(t)$ – в области частоты $\omega _c$, значение которой обычно намного 
	превышает значение наивысшей частоты модуляции. Преобразование спектра при 
	демодуляции возможно только в устройствах, выполняющих нелинейное или 
	параметрическое преобразование входного сигнала $U_c(t)$. В качестве подобного 
	устройства может быть использован диод или транзистор.
	
	\includeimage
		{The_principle_of_operation_of_the_amplitude_detector}
		{f} % Обтекание (без обтекания)
		{h} % Положение рисунка (см. figure из пакета float)
		{0.75\textwidth} % Ширина рисунка
		{Принцип работы детектора.} % Подпись рисунка
	
	При использовании нелинейного устройства, обладающего квадратичной 
	вольт‐амперной характеристикой, выходной ток имеет вид
	\begin{center}
		$i = B u_c^2(t) = Bu_a^2(t)[0.5 + 0.5 cos(2 \omega _c t)],$
	\end{center}
	
	где $В$ - постоянный коэффициент. После устранения фильтром низких частот
	(ФНЧ) составляющей с частотой $2\omega_c$ получим
	\begin{center}
		$i = 0.5 B U_c^2[1 + 2m_a x(t) + m_a^2 x^2(t)].$
	\end{center}
\indent	В этом токе содержится составляющая вида $BU_c^2m_ax(t)$, пропорциональная
	передаваемому сообщению, а также составляющая $0.5BU_c^2m_a^2x^2(t)$
	, которая определяет степень нелинейных искажений модулирующего сообщения
	$x(t)$. \\
	Параметрическое преобразование осуществляется путем умножения $U_c(t)$ на
	опорное колебание, имеющее вид 
	\begin{center}
		$U_0(t) = U_0 cos(\omega_c t).$
	\end{center}
	В этом случае результат перемножения определяется следующим выражением
	\begin{center}
		$U_c(t)U_0(t) = u_a(t)U_0[0.5 + 0.5cos(2\omega_c t)].$
	\end{center}
\indent	Составляющая с частотой $2\omega_c$ устраняется ФНЧ и в результате 
	формируется низкочастотный сигнал вида $0.5 U_0 u_a(t)$. Отделяя постоянную 
	составляющую $0.5 U_0 U_c$, например, при помощи разделительного конденсатора, 
	получаем сигнал вида $0.5 U_0 U_c max(t)$, форма которого определяется 
	передаваемым сообщением $x(t)$. \\
\indent	Амплитудный детектор, выполняемый по микрополосковой технологии, состоит из
	элемента связи с СВЧ‐трактом (согласующего устройства), диода, ФНЧ и вывода
	сигнала на НЧ. \\
\indent	Значение тока через диод $i_g$ для режима покоя $(u_c(t)=0)$ может быть 
	найдено из уравнений
	\begin{equation*}
		\begin{cases}
			 i_g = f(U_g)\\
			 i_g = - \frac{u_\text{вых}}{R_\text{н}}
		\end{cases}
	\end{equation*}
\indent	где $U_g$ – напряжение на диоде. Первое уравнение является уравнением 		
	вольтамперной характеристики (ВАХ) диода как безынерционного нелинейного 
	элемента. Из‐за нелинейного характера ВАХ, форма тока через диод $i_g$ при 
	синусоидальной форме сигнала $U_c(t)$ не является синусоидальной. В составе 
	тока появляется постоянная составляющая, которая, протекая по резистору 
	$R_\text{н}$, создает падение напряжения $U_g$, смещающая положение рабочей 
	точки. При увеличении амплитуды входного напряжения смещение рабочей точки 
	увеличивается, и ток через диод будет приближаться по форме к однополярным 
	импульсам, открывающим диод при положительных значениях входного напряжения.
	 
	\includeimage
		{Graph_with_the_detector_operation_principle}
		{f} % Обтекание (без обтекания)
		{h} % Положение рисунка (см. figure из пакета float)
		{0.75\textwidth} % Ширина рисунка
		{Детектирование АМ сигналов.} % Подпись рисунка
	
\indent	На~рисунке~\ref{img:Graph_with_the_detector_operation_principle} приведены 
	формы
	напряжений и токов на входе детектора для двух случаев, когда амплитуды входных
	сигналов удовлетворяют неравенству $U_c(1) < U_c(2)$. Тогда постоянные 
	составляющие напряжений $U_c(1) < U_c(2)$ и $I_c(1) < I_c(2)$. На этом же 
	рисунке изображена зависимость $i_g = f(t)$. \\
\indent	Вольтамперная характеристика диода в широком диапазоне токов достаточно 
	точно аппроксимируется экспоненциальной зависимостью
	\begin{center}
		$i_g = I_\text{об} (e^{u_g(t)/\phi_T} - 1),$
	\end{center}

\indent где $I_\text{об}$ – абсолютное значение величины обратного тока диода,
	$\phi_T$ – температурный потенциал, равный при $T = 293^{\circ}$ примерно 26
	мВ. Из этой зависимости следует, что \\
\indent	‐ с увеличением $R_\text{н}$ увеличивается крутизна детекторной
	характеристики, \\
\indent	‐ с увеличением уровня сигнала снижается нелинейность детекторной 
	характеристики. \\
\indent	Из этого следует, что диодный детектор работает в двух режимах когда на вход
	поступает «слабый» сигнал и когда – «сильный». В режиме «слабого» сигнала
	характеристика диода аппроксимируется квадратичной зависимостью, в режими 
	сильных токов – линейной зависимостью.
	
	
	\chapter{Синтез топологии платы в программе AWR Design Environment 2009.}
	
	Амплитудный детектор, выполняемый по микрополосковой технологии, состоит из 
	копланарной линии, диода, фильтра и вывода сигнала на НЧ. Т.к. информационный 
	сигнал имеет частоту в полосе от 7700 МГц до 8300МГц, как следует из условия, 
	нужно использовать полосно‐пропускной фильтр. \\
\indent	Параметры диода и подложки выбираются по техническому заданию. Подложка – 
	поликор $(e_r = 9.8; T_{and} = 0.001)$. Диод – 3А206 А‐6.
	
	% Вставить параметры диода
	
\indent	Рассчитаем детктор в программе AWR Design.

\includeimage
	{Circuit_on_the_coplanar_line}
	{f} % Обтекание (без обтекания)
	{H} % Положение рисунка (см. figure из пакета float)
	{0.75\textwidth} % Ширина рисунка
	{Схема микрополоскового детектора на копланарной линии.} % Подпись рисунка

	Для расчёта копланарной линии с R=50 Ом прибегнем к помощи инструмента TXLine.
	
\includeimage
	{TXLine_coplanar_line}
	{f} % Обтекание (без обтекания)
	{H} % Положение рисунка (см. figure из пакета float)
	{0.75\textwidth} % Ширина рисунка
	{Результат расчёта копланарной линии.} % Подпись рисунка

	Заметим, что результаты вычисления копланарной линии не удовлетворяют нас по
	предельным возможностям технологий. Введём изменение – уменьшим толщину
	проводящего слоя до 10 мкм.

\includeimage
	{TXLine_coplanar_line_new}
	{f} % Обтекание (без обтекания)
	{h} % Положение рисунка (см. figure из пакета float)
	{0.75\textwidth} % Ширина рисунка
	{Откорректированные результаты расчёта.} % Подпись рисунка
	
	Введём полученные значения в схему.
	
\includeimage
	{Circuit_on_the_coplanar_line_new}
	{f} % Обтекание (без обтекания)
	{h} % Положение рисунка (см. figure из пакета float)
	{0.75\textwidth} % Ширина рисунка
	{} % Подпись рисунка

	Построим график передачи мощности.
	
\includeimage
	{}
	{f} % Обтекание (без обтекания)
	{h} % Положение рисунка (см. figure из пакета float)
	{0.75\textwidth} % Ширина рисунка
	{} % Подпись рисунка

\includeimage
	{}
	{f} % Обтекание (без обтекания)
	{h} % Положение рисунка (см. figure из пакета float)
	{0.75\textwidth} % Ширина рисунка
	{} % Подпись рисунка

\includeimage
	{}
	{f} % Обтекание (без обтекания)
	{h} % Положение рисунка (см. figure из пакета float)
	{0.75\textwidth} % Ширина рисунка
	{} % Подпись рисунка

\includeimage
	{}
	{f} % Обтекание (без обтекания)
	{h} % Положение рисунка (см. figure из пакета float)
	{0.75\textwidth} % Ширина рисунка
	{} % Подпись рисунка

	\chapter{Расчет рабочих параметров.}
	
	\chapter{Анализ влияния ПД на параметры диода на рабочие характеристики устройства.}
	
	\chapter{Технология изготовления.} 
	
	
	\begin{thebibliography}{}
		\bibitem{litlink1} Твердотельные устройства в технике связи/ Л.Г. Гассанов и др. – М.: Радио и связь, 1988.
		\bibitem{litlink2} Малорацкий Л.Г., Микроминиатюризация элементов и устройств СВЧ., М., «Сов. Радио», 1976.
		\bibitem{litlink3} Бушминский И.П., Гудков А.Г., Дергачев В.Ф. Конструкторское проектирование микросхем СВЧ: Учеб. пособие. М.: Изд-во МГТУ им. Н.Э.Баумана, 1991, 225 с.
		\bibitem{litlink4} Справочник по расчету и конструированию СВЧ полосковых устройств / С.И.Бахарев, В.И.Вольман и др.: Под. ред. В.И.Вольмана. М.: Радио и связь, 1982, 328 с.
		\bibitem{litlink5} Полупроводниковые приборы. Сверхвысокочастотные диоды. Справочник /Б.А. Наливайко и др. Под ред. Б.А. Наливайко. – Томск: МГП «РАСКО», 1992.
		\bibitem{litlink6} ГОСТ 2.734 – 68. Обозначения условные графические в схемах. Линии сверхвысокой частоты и их элементы.
		\bibitem{litlink7} ОСТ 107.750 878.002 – 87   Технология изготовления толстопленочных плат.
		\bibitem{litlink8} \href{http://mart7157.narod.ru/voise_10.html}{Первая картинка}
	\end{thebibliography}
	
%	На~рисунке~\ref{img:BCD} пример семисегментного индикатора.
%	\includeimage
%	{BCD} % Имя файла без расширения (файл должен быть расположен в директории inc/img/)
%	{f} % Обтекание (без обтекания)
%	{h} % Положение рисунка (см. figure из пакета float)
%	{0.25\textwidth} % Ширина рисунка
%	{Семисегментный индикатор} % Подпись рисунка 	
	
\end{document}





























