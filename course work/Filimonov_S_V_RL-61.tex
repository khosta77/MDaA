\documentclass{bmstu}
\usepackage{multirow}
%\usepackage{karnaugh-map}
\usepackage{tikz}
\usepackage{float}

\usetikzlibrary{karnaugh}

\begin{document}

	\makecourseworktitle
	{Радиоэлектроника и лазерная техника (РЛ)} % Название факультета
	{Технология приборостроения (РЛ6)} % Название кафедры
	{Проектирование конструкции микрополоскового детектора} % Тема работы
	{Филимонов~С.~В./РЛ6-61} % Номер группы/ФИО студента (если авторов несколько, их необходимо разделить запятой)
	{Федоркова Н.В.} % ФИО научного руководителя
	{}
	
	\tableofcontents
	
	\chapter{Принцип работы устройства.}
	
	Амплитудный детектор – это устройство, предназначенное для получения на
	выходе напряжения, изменяющегося в соответствии с законом модуляции
	амплитуды входного гармонического сигнала. Процесс детектирования
	амплитудно-модулированных (АМ) сигналов вида
	
	\begin{center}
		 $U_c(t) = u_a(t) \cdot cos(\omega _c \cdot t), \text{где } u_a(t)=U_c[1 + max(t)],$
	\end{center}
	
	где $m_a$ < 1 - коэффициент глубины модуляции, $U_c$ амплитуда несущего 
	колебания с частотой $\omega _c$, заключается в воспроизведении модулирующего 
	сообщения $x(t)$ с наименьшими искажениями. Спектр сообщения $x(t)$ 
	сосредоточен в области низких частот (частот модуляции), а спектр сигнала 
	$U_c(t)$ – в области частоты $\omega _c$, значение которой обычно намного 
	превышает значение наивысшей частоты модуляции. Преобразование спектра при 
	демодуляции возможно только в устройствах, выполняющих нелинейное или 
	параметрическое преобразование входного сигнала $U_c(t)$. В качестве подобного 
	устройства может быть использован диод или транзистор.
	
	\includeimage
		{The_principle_of_operation_of_the_amplitude_detector}
		{f} % Обтекание (без обтекания)
		{h} % Положение рисунка (см. figure из пакета float)
		{0.75\textwidth} % Ширина рисунка
		{Принцип работы детектора.} % Подпись рисунка
	
	При использовании нелинейного устройства, обладающего квадратичной 
	вольт‐амперной характеристикой, выходной ток имеет вид
	\begin{center}
		$i = B u_c^2(t) = Bu_a^2(t)[0.5 + 0.5 cos(2 \omega _c t)],$
	\end{center}
	
	где $В$ - постоянный коэффициент. После устранения фильтром низких частот
	(ФНЧ) составляющей с частотой $2\omega_c$ получим
	\begin{center}
		$i = 0.5 B U_c^2[1 + 2m_a x(t) + m_a^2 x^2(t)].$
	\end{center}
\indent	В этом токе содержится составляющая вида $BU_c^2m_ax(t)$, пропорциональная
	передаваемому сообщению, а также составляющая $0.5BU_c^2m_a^2x^2(t)$
	, которая определяет степень нелинейных искажений модулирующего сообщения
	$x(t)$. \\
	Параметрическое преобразование осуществляется путем умножения $U_c(t)$ на
	опорное колебание, имеющее вид 
	\begin{center}
		$U_0(t) = U_0 cos(\omega_c t).$
	\end{center}
	В этом случае результат перемножения определяется следующим выражением
	\begin{center}
		$U_c(t)U_0(t) = u_a(t)U_0[0.5 + 0.5cos(2\omega_c t)].$
	\end{center}
\indent	Составляющая с частотой $2\omega_c$ устраняется ФНЧ и в результате 
	формируется низкочастотный сигнал вида $0.5 U_0 u_a(t)$. Отделяя постоянную 
	составляющую $0.5 U_0 U_c$, например, при помощи разделительного конденсатора, 
	получаем сигнал вида $0.5 U_0 U_c max(t)$, форма которого определяется 
	передаваемым сообщением $x(t)$. \\
\indent	Амплитудный детектор, выполняемый по микрополосковой технологии, состоит из
	элемента связи с СВЧ‐трактом (согласующего устройства), диода, ФНЧ и вывода
	сигнала на НЧ. \\
\indent	Значение тока через диод $i_g$ для режима покоя $(u_c(t)=0)$ может быть 
	найдено из уравнений
	\begin{equation*}
		\begin{cases}
			 i_g = f(U_g)\\
			 i_g = - \frac{u_\text{вых}}{R_\text{н}}
		\end{cases}
	\end{equation*}
\indent	где $U_g$ – напряжение на диоде. Первое уравнение является уравнением 		
	вольтамперной характеристики (ВАХ) диода как безынерционного нелинейного 
	элемента. Из‐за нелинейного характера ВАХ, форма тока через диод $i_g$ при 
	синусоидальной форме сигнала $U_c(t)$ не является синусоидальной. В составе 
	тока появляется постоянная составляющая, которая, протекая по резистору 
	$R_\text{н}$, создает падение напряжения $U_g$, смещающая положение рабочей 
	точки. При увеличении амплитуды входного напряжения смещение рабочей точки 
	увеличивается, и ток через диод будет приближаться по форме к однополярным 
	импульсам, открывающим диод при положительных значениях входного напряжения.
	 
	\includeimage
		{Graph_teory}
		{f} % Обтекание (без обтекания)
		{h} % Положение рисунка (см. figure из пакета float)
		{0.75\textwidth} % Ширина рисунка
		{Детектирование АМ сигналов.} % Подпись рисунка
	
\indent	На~рисунке~\ref{img:Graph_teory} приведены 
	формы
	напряжений и токов на входе детектора для двух случаев, когда амплитуды входных
	сигналов удовлетворяют неравенству $U_c(1) < U_c(2)$. Тогда постоянные 
	составляющие напряжений $U_c(1) < U_c(2)$ и $I_c(1) < I_c(2)$. На этом же 
	рисунке изображена зависимость $i_g = f(t)$. \\
\indent	Вольтамперная характеристика диода в широком диапазоне токов достаточно 
	точно аппроксимируется экспоненциальной зависимостью
	\begin{center}
		$i_g = I_\text{об} (e^{u_g(t)/\phi_T} - 1),$
	\end{center}

\indent где $I_\text{об}$ – абсолютное значение величины обратного тока диода,
	$\phi_T$ – температурный потенциал, равный при $T = 293^{\circ}$ примерно 26
	мВ. Из этой зависимости следует, что \\
\indent	‐ с увеличением $R_\text{н}$ увеличивается крутизна детекторной
	характеристики, \\
\indent	‐ с увеличением уровня сигнала снижается нелинейность детекторной 
	характеристики. \\
\indent	Из этого следует, что диодный детектор работает в двух режимах когда на вход
	поступает «слабый» сигнал и когда – «сильный». В режиме «слабого» сигнала
	характеристика диода аппроксимируется квадратичной зависимостью, в режими 
	сильных токов – линейной зависимостью.
	
	
	\chapter{Синтез топологии платы в программе AWR Design Environment.}
	
	Амплитудный детектор, выполняемый по микрополосковой технологии, состоит из 
	копланарной линии, диода, фильтра и вывода сигнала на НЧ. Т.к. информационный 
	сигнал имеет частоту в полосе от 7700 МГц до 8300МГц, как следует из условия, 
	нужно использовать полосно‐пропускной фильтр. \\
\indent	Параметры диода и подложки выбираются по техническому заданию. Подложка – 
	поликор $(e_r = 9.8; T_{and} = 0.001)$. Диод – 3А206А-6 \cite{9}.
	
\includeimage
	{diod_3a206_6}
	{f} % Обтекание (без обтекания)
	{H} % Положение рисунка (см. figure из пакета float)
	{1.0\textwidth} % Ширина рисунка
	{Параметры диода 3A206A-6.} % Подпись рисунка
	
\indent	Рассчитаем детктор в программе AWR Design.

%\includeimage
%	{Circuit_on_the_coplanar_line_new}
%	{f} % Обтекание (без обтекания)
%	{H} % Положение рисунка (см. figure из пакета float)
%	{1.0\textwidth} % Ширина рисунка
%	{Схема микрополоскового детектора на копланарной линии.} % Подпись рисунка

	Для расчёта копланарной линии с R=50 Ом прибегнем к помощи инструмента TXLine.
	
\includeimage
	{TXLine_coplanar_line}
	{f} % Обтекание (без обтекания)
	{H} % Положение рисунка (см. figure из пакета float)
	{1.0\textwidth} % Ширина рисунка
	{Результат расчёта копланарной линии.} % Подпись рисунка

	Заметим, что результаты вычисления копланарной линии не удовлетворяют нас по
	предельным возможностям технологий. Введём изменение – уменьшим толщину
	проводящего слоя до 10 мкм.

\includeimage
	{TXLine_coplanar_line_new}
	{f} % Обтекание (без обтекания)
	{h} % Положение рисунка (см. figure из пакета float)
	{1.0\textwidth} % Ширина рисунка
	{Откорректированные результаты расчёта.} % Подпись рисунка
	
	Введём полученные значения в схему.
	
\includeimage
	{Circuit_on_the_coplanar_line_new}
	{f} % Обтекание (без обтекания)
	{H} % Положение рисунка (см. figure из пакета float)
	{1.0\textwidth} % Ширина рисунка
	{Схема с обновленными параметрами.} % Подпись рисунка

	Построим график передачи мощности.
	
\includeimage
	{Graph_LSSnm_measurement}
	{f} % Обтекание (без обтекания)
	{H} % Положение рисунка (см. figure из пакета float)
	{0.90\textwidth} % Ширина рисунка
	{Параметры графика.} % Подпись рисунка

	В результате получим:
% график LSSnm
\includeimage
	{Graph_LSSnm}
	{f} % Обтекание (без обтекания)
	{H} % Положение рисунка (см. figure из пакета float)
	{1.0\textwidth} % Ширина рисунка
	{График LSSnm.} % Подпись рисунка
	
	Из графика не понятно где происходит переход, зададим границы от -10 до -3 дБ.
	В результате получим:
	
\includeimage
	{Graph_LSSnm_new}
	{f} % Обтекание (без обтекания)
	{H} % Положение рисунка (см. figure из пакета float)
	{0.8\textwidth} % Ширина рисунка
	{График LSSnm.} % Подпись рисунка
	
	Исследуя зависимость от тока, согласно полученным данным характеристика имеет линейный характер на участке от -10 дБ до -3 дБ.
	
	\chapter{Расчет рабочих параметров.}
	
	\indent	Параметры диода и подложки выбираются по техническому заданию. Подложка – 
	поликор $(e_r = 9.8; T_{and} = 0.001)$. Диод – 3А206А-6 \cite{9}.
	
	\includeimage
	{diod_3a206_6}
	{f} % Обтекание (без обтекания)
	{H} % Положение рисунка (см. figure из пакета float)
	{1.0\textwidth} % Ширина рисунка
	{Параметры диода 3A206A-6.} % Подпись рисунка
	
	\includeimage
	{detector_circuit}
	{f} % Обтекание (без обтекания)
	{H} % Положение рисунка (см. figure из пакета float)
	{1.0\textwidth} % Ширина рисунка
	{Принципиальная схема микрополосового детектора.} % Подпись рисунка
	
	\includeimage
	{detector_IC}
	{f} % Обтекание (без обтекания)
	{H} % Положение рисунка (см. figure из пакета float)
	{1.0\textwidth} % Ширина рисунка
	{Чертеж микрополосового детектора.} % Подпись рисунка
	
	\chapter{Анализ влияния ПД на параметры диода и рабочие характеристики устройства.}
	
	Проводить анализ влияние ПД на параметры диода и на рабочие характеристики устройства не нужно.
	
	\chapter{Технология изготовления.} 
	
	   ОТРАСЛЕВОЙ   СТАНДАРТ \\
	ТЕХНОЛОГИЯ ИЗГОТОВЛЕНИЯ тонкопленочных плат     ОСТ 107.750878.001-87\\
	Общие требования \\
	1. ОСНОВНЫЕ ПОЛОЖЕНИЯ \\
	1.1. Общие требования к технологии изготовления плат направлены на обеспечение их исполнения в соответствии с требованиями и нормами конструирования согласно ОСТ 4Г 0.010.224-82 и ОСТ 107.460084.200-88. \\
	1.2. Общие требования настоящего стандарта предназначены для разработки и внедрения прогрессивной технологии изготовления плат с использованием перспективного технологического оборудования. \\
	1.3. В качестве основы при разработке технологии изготовления конкретных типов плат наряду с обязательными требованиями настоящего стандарта могут быть приняты не противоречащие им дополнительные требования к технологии, необходимые для изготовления плат в соответствии с требованиями конструкторской документации (КД).
	При необходимости разработки новой технологии -изготовления плат, не предусмотренной настоящим стандартом, требования к ней подлежат обязательному согласованию с предприятием-разработчиком стандарта. \\
	
	2. ОСНОВНЫЕ ТРЕБОВАНИЯ  \\
	2.1. Требования к качеству изготовления плат \\
	2.1.1. Качество изготовления плат должно обеспечивать их функционирование в составе РЭС в течение срока минимальной \\
	наработки, установленного техническими условиями (ТУ) на РЭС. \\
	2.1.2. Качество изготовления плат должно обеспечивать соответствие их конструкции требованиям КД на плату  ОСТ 107.750871.001 -86  и ОСТ  107.750871.002-90 \\
	2.1.3. Качество изготовления плат в зависимости от стадии изготовления следует оценивать по критериям, установленным настоящим стандартом , ОСТ  107.750871.001-86 ,
	ОСТ 107.750871.002-90 ,а также КД на плату. \\
	2.2. Требования к материалам \\
	2.2.1. Материалы должны обеспечивать получение технических характеристик плат, соответствующих требованиям ОСТ 4Г 0.010.224-82, ОСТ 107.460084.200-88 и КД на плату. \\
	Перечень материалов, рекомендуемых для изготовления плат, приведен в приложении 1. \\
	2.2.2. Материалы должны обеспечивать качество изготовления плат в соответствии с требованиями подраздела 2.1. \\
	2.2.3. Материалы должны иметь утвержденные ТУ и сертификаты, подтверждающие их приемку отделом технического контроля (ОТК) завода-изготовителя, и дату изготовления. \\
	2.2.4. Материалы должны использоваться для изготовления плат в течение срока их годности, указанного в ТУ. \\
	2.2.5. Материалы должны иметь технические характеристики, соответствующие указанным в ТУ. \\
	
	ОСТ 107.750878.001-87 С. 3 \\
	
	2.2.6. Требования к технологическим газам и воде по  ОСТ 11050.003-83 и ОСТ 11029.003-80 \\
	соответственно с учётом условий производства тонкоплёночных плат. \\
	
	2.3. Требования к технологическому оборудованию, средствам контроля и измерений  \\
	2.3.1. Оборудование должно обеспечивать получение технических характеристик плат, соответствующих требованиям ОСТ 107.460084.200-88 или ОСТ 4Г 0.010,224-82, а также требованиям КД на плату. \\
	Перечень технологического оборудования, средств контроля и измерений, рекомендуемых для изготовления плат, приведен в приложении 2. \\
	2.3.2. Оборудование должно обеспечивать качество изготовления плат в соответствии с требованиями подраздела 2.1. \\
	2.3.3. Оборудование, средства контроля и измерений должны иметь технический паспорт или заменяющий его документа, а также полный комплект эксплуатационной документации. \\
	2.3.4. Оборудование должно обеспечивать технологические режимы обработки в соответствии с требованиями технологической документации на плату и раздела 5. \\
	2.3.5. При необходимости могут быть разработаны, изготовлены и аттестованы нестандартизованные средства  измерений в соответствии с требованиями ГОСТ 8.326-89 и РД 107.80013-91. \\
	2.3.6. Средства  измерений должны обеспечивать    заданную точность  измерений. \\
	2.3.7. Средства  измерений должны проходить поверку согласно ГОСТ 8.513-84. \\
	2.3.8. Подготовку к работе оборудования, средств контроля и измерений и их эксплуатацию при изготовлении плат производить в соответствии с их эксплуатационной документацией. \\
	
	
	2.4. Требования к условиям изготовления плат \\
	2.4.1. Условия изготовления плат должны обеспечивать получение технических характеристик плат, установленных КД на платы, и качества их изготовления, соответствующего требованиям подраздела 2.1. \\
	2.4.2. Технологические помещения для изготовления плат должны соответствовать требованиям  РД 107.460093.002-90 \\
	2.4.3. Технологическая одежда должна соответствовать требованиям  ОСТ  107.460082.001-86,  а  принадлежности – РД 11 14.3316-89. \\
	
	С. 4 ОСТ 107.750878.001-87 \\
	2.4.4. Наружные поверхности установок, столов, наружные и внутренние поверхности вытяжных шкафов протирать влажной, а затем сухой салфеткой из холстопрошивного обтирочного полотна; рабочие поверхности и детали технологического оборудования, средств контроля и измерений, пинцеты, резиновые перчатки и напальчники протирать салфетками из отбеленной хлопчатобумажной бязи и отбеленного батиста или ватно-батисто-выми тампонами, смоченными этиловым ректификованным техническим спиртом.
	Протирку выполнять перед началом работы и по мере загрязнения, но не реже двух раз в смену. \\
	2.5. Требования к условиям, срокам межоперационного хранения и транспортирования плат и подложек \\
	2.5.1. Условия и допустимые сроки межоперационного хранения, а также транспортирование плат и подложек должны обеспечивать сохранение качества изготовления плат, соответствующего требованиям подраздела 2.1. \\
	2.5.2. Допустимые сроки межоперационного хранения должны соответствовать интервалам времени, указанным в разделе 5. \\
	2.5.3. Межоперационное транспортирование подложек и плат производить в закрытой таре, исключающей механические повреждения подложек и плат. \\
	2.5.4. Внутриоперационное транспортирование подложек и плат вручную производить с использованием пинцета, напальчников или перчаток. \\
	2.5.5. Межоперационное хранение подложек и плат в процессе их изготовления осуществлять в таре, помещенной в устройство для хранения деталей в защитной среде. \\
	2.5.6. Изготовленные платы хранить в таре, которая обеспечивает защиту от попадания пыли, влаги и других загрязнений на поверхности плат, а также от механических повреждении плат при транспортировании и хранении. \\
	2.5.7. Хранение изготовленных плат должно происходить в сухих отапливаемых помещениях при температуре воздуха от 15 до 35 °С, относительной влажности (65 ±15) \% и отсутствии агрессивных паров. \\
	2.5.8. Срок хранения плат после их изготовления и до герметизации в изделиях, которые этими платами комплектуются, равен шести месяцам. \\
	
	5. ТРЕБОВАНИЯ К ТЕХНОЛОГИИ \\
	5.1. Требования к изготовлению плат \\
	5.1.1. Изготовление плат должно быть основано на комплексном использовании следующих технологических методов: \\
	1) очистки; \\
	2) вакуумного напыления; \\
	3) изготовления фоторезистивной маски; \\
	4) химического травления; \\
	5) ионного травления; \\
	6) плазмохимической обработки; \\
	7) химического осаждения; \\
	8) электрохимического осаждения; \\
	9) электрохимического анодного окисления; \\
	10) термической обработки; \\
	11) ультразвуковой обработки; \\
	12) лужения; \\
	13) лазерной обработки; \\
	14) электроэрозионной обработки; \\
	15) механической обработки. \\
	Состав комплекса технологических методов для изготовления платы должен обеспечивать выполнение требований КД на плату. \\
	5.1.2. Технология изготовления платы в зависимости от требований КД должна обеспечивать изготовление тонкопленочных резисторов, проводниковых элементов, обкладок и диэлектриков конденсаторов, межуровневой изоляции, защитного слоя, переходных контактных элементов. Рекомендуемые примеры схем изготовления плат приведены в приложении 3. \\
	5.1.3. Изготовление рисунка тонкопленочных элементов плат должно быть основано на методе фотолитографии с использованием химического или ионного травления.
	Допускается по решению главного конструктора изделия, в котором используются платы, изготавливать платы с применением биметаллических масок. \\
	5.1.4. Технология изготовления плат должна обеспечивать получение толщин нанесенных слоев и их электрических параметров, установленных КД на плату. \\
	5.1.5. Технология изготовления плат должна обеспечивать прочное сцепление слоев (структур) с поверхностью подложки и между собой. \\
	Требования по прочности сцепления слоев (структур) с подложкой и между собой для микрополосковых плат—по ОСТ 107.750871.001-86.  \\
	5.1.6. Технология изготовления должна обеспечивать способность плат к пайке нанесением покрытий, предусмотренных КД, на проводниковые элементы. Лужение поверхностей проводниковых элементов плат производить в соответствии с требованиями ОСТ 107.460091.004-88. Удаление остатков флюсов после лужения производить с применением моющих средств по ОСТ 4Г 0.029.233-84.   \\   
	5.1.7. Перед началом каждой технологической операции протирать резиновые перчатки, напальчники, инструмент, детали оснастки и приспособлений хлопчатобумажным батистом или бязью, смоченными этиловым техническим ректифицированным спиртом. \\
	5.1.8. В процессе изготовления подложки и платы следует брать за боковые (торцевые) поверхности пинцетом или руками, защищенными резиновыми перчатками или напальчниками. \\
	5.2. Требования к очистке \\
	5.2.1. Очистка подложек должна включать: \\
	1) обработку моющими средствами; \\
	2) промывку; \\
	3) сушку. \\
	5.2.2. Выбор моющих средств для обработки подложек, за исключением полиимидных, производить по ОСТ 4Г 0.029.233-84. \\
	Обработку подложек из керамики и феррита производить с использованием ультразвукового воздействия на  частоте не менее 18 кГц. \\
	5.2.3. Обработку полиимидных подложек производить в хромовой смеси. Рекомендуемый состав хромовой смеси приведен в приложении 4. \\
	5.2.4. Промывку подложек следует производить в проточной дистиллированной или деионизованной воде. \\
	5.2.5. Сушку подложек, за  исключением  полиимидных, производить при температуре (120±:5) °С в течение (15±5) мин. \\
	Допускается производить сушку подложек на центрифуге при использовании специальных линий очистки подложек, в которых предусмотрена такая сушка. \\
	5.2.6. Сушку (отжиг) полиимидных подложек производить в среде инертного газа при температуре не менее 200 °С в течение (60±5) мин. \\
	5.2.7. Поверхность подложки, прошедшей очистку, должна быть чистой, без подтеков, пятен и инородных материалов. \\
	5.2.8. Очистку подложек следует производить непосредственно перед нанесением на подложку слоев. В обоснованных случаях допускается перерыв между окончанием очистки и началом нанесения слоев, который не должен превышать 6 ч при хранении подложек в эксикаторе с силикагелем или 24 ч при хранении в шкафу с  защитной  средой. \\
	Использование для нанесения слоев подложек, хранение которых после очистки превысило установленный срок, допускается только после повторной их очистки в соответствии с требованиями настоящего стандарта. \\
	5.2.9. Очистка навесок напыляемых материалов должна включать: \\
	1) обработку моющими средствами; \\
	2) травление; \\
	3) осветление; \\
	4) промывку; \\
	5) сушку. \\
	Очистку навесок, подготовка которых к напылению гарантирована ТУ на навески, не производить. \\
	5.2.10. Обработку навесок моющими средствами производить с использованием ультразвукового воздействия на частоте не менее 18 кГц. Выбор моющих средств производить но ОСТ 4Г 0.029.233-84. \\
	Растворы для травления и осветления навесок из различных материалов приведены в приложении 5. \\
	5.2.11. Промывку навесок следует производить в проточной дистиллированной или деионизованной воде после обработки моющими средствами, после травления и после осветления.
	5.2.12. Сушку навесок производить в среде инертного газа или обеспыленного воздуха при температуре (100 ±5) °С в течение (20±5) мин. \\
	5.2.13. Поверхность навесок, прошедших очистку, должна быть без пятен, подтеков и инородных материалов. \\
	5.2.14. Очистку навесок следует производить не ранее чем за 5 суток до использования их для напыления слоев. После очистки навески сразу должны быть помещены в эксикатор с силикагелем или шкаф с защитной атмосферой. 5.2.15. Очистка испарителей должна включать: \\
	1) обработку моющими средствами; \\
	2) травление; \\
	3) промывку; \\
	4) сушку; \\
	5) отжиг. \\
	5.2.16. Обработку испарителей моющими средствами производить с использованием ультразвукового воздействия на частоте не менее 18 кГц. Выбор моющих средств производить по ОСТ 4Г 0.029.233-84. \\
	Растворы для травления испарителей из различных материалов, а также для осветления медных контактов испарителей приведены в приложении 6.
	5.2.17. Промывку испарителей следует производить в проточной дистиллированной или деионизованной воде после обработки моющими средствами, после травления и после осветления медных контактов испарителей. \\
	5.2.18. Сушку испарителей производить в среде инертного газа или обеспыленного воздуха при температуре (120±10)°С в течение (20:4:5) мин. \\
	5.2.19. После сушки испарители сразу должны быть помещены в эксикатор с силикагелем или в шкаф с защитной атмосферой. Разрыв между сушкой и отжигом испарителей не должен быть более 5 сут. \\
	5.2.20. Поверхность испарителей, поступающих на отжиг, должна быть без пятен, подтеков и инородных материалов. \\
	5.2.21. Отжиг испарителей производить на их рабочих позициях в установках вакуумного напыления. Режим отжига следует подбирать из расчета достижения температуры испа¬рителя не менее их рабочей температуры во время напыления слоев. Время отжига определяется достижением в вакуумной камере остаточного давления не более 2.7*10-3  Па (2*10-5  мм. рт.ст.). \\
	5.2.22. Механическую очистку подколпачного устройства установок вакуумного напыления (без его разборки) с удалением продуктов очистки пылесосом производить, руководствуясь инструкцией по эксплуатации установки, не реже , чем через каждые 15 напылений. Очистку съемных деталей подколпачного устройства химическими методами производить не реже, чем через каждые 70 напылений. Составы растворов для очистки следует подбирать с учетом материалов напыленных слоев. \\
	5.2.23. Внутренняя поверхность рабочей камеры установки вакуумного напыления и поверхности деталей подколпачного устройства, подготовленные к напылению, должны быть чистыми,  не содержать следов  масел,  используемых в откачных средствах, легко отделяемых остатков материалов от предыдущих напылений, а также загрязнений посторонними материалами. \\
	5.3. Требования к вакуумному напылению и плазмохимической обработке \\
	5.3.1. Напыление слоев методом термического испарения производить при остаточном давлении в камере не более 1.33*10-3 Па (1*10-5 мм рт.ст.). Откачку камеры производить с обязательным использованием ловушки, охлаждаемой жидким азотом. Допускается работа без применения азотной ловушки при использовании в диффузионных насосах рабочих жидкостей с давлением пара при 20 °С не более 1,3*10-8  Па  (1*10-10  мм рт. ст.) или других средств безмасляной откачки. \\
	5.3.2. Напыление слоев ионно-плазменным методом в установках с трехэлектродной системой распыления производить после предварительной откачки вакуумной камеры до остаточного давления не более 1,33*10-3  Па (1*10-5   мм рт.ст.) в дуговом разряде при давлении аргона в вакуумной камере не более 2.66*10-1  Па (2*10-3  мм рт. cт.). Откачку камеры производить с обязательным использованием ловушки, охлаждаемой жидким азотом. Допускается работа без применения азотной ловушки при использовании в диффузионных насосах рабочих жидкостей с давлением пара при 20 °С не более 1,3*10-8  Па (1*10-10  мм рт. ст.) или других средств безмасляной откачки. Напыление слоев производить с предварительной тренировкой мишени. Продолжительность тренировки мишени, с которой ранее производились напыления, не менее 10 мин при подаче на мишень мощности не менее необходимой для последующего напыления.  \\
	5.3.3. Напыление резистивных слоев методом термического испарения порошковых материалов производить взрывным испарением с помощью вибропитателя. Для получения резистивных слоев с параметрами, установленными ТУ на порошковые материалы, напыление необходимо производить с соблюдением рекомендаций этих ТУ. \\
	5.3.4. Напыление резистивных слоев ионно-плазменным методом необходимо производить со скоростями осаждения пленки, обеспечивающими получение параметров резистивных слоев, установленных ТУ на мишени. \\
	5.3.5. Термообработку напыленных  резистивных слоев производить в режимах, рекомендованных ТУ на резистивный материал. \\
	5.3.6. Достижение установленной величины удельного поверхностного сопротивления резистивного слоя в процессе напыления следует контролировать с помощью контрольной платы или подложки (далее—«свидетеля»). \\
	5.3.7. При напылении проводниковых слоев методом термического испарения в качестве адгезионного слоя (далее — подслоя) следует предварительно напылять какой-либо из следующих материалов: хром марки ЭРХ, нихром марки Х20Н80, ванадий марок ВнПл-1 и ВнПр-1, титан марки ВТ1-0. \\
	Допускается не -производить напыление подслоя при напылении слоя алюминия для изготовления проводников второго уровня двухуровневых плат, а также в тех случаях, когда материал резистивного слоя платы, напыляемый в едином вакуумном цикле с проводниковым слоем, является тем же, что и материал подслоя. \\
	Напыление подслоя, не являющегося одновременно резистивным слоем, рекомендуется производить до достижения удельного поверхностного сопротивления (200 ±50) Ом на квадрат, которое контролируется по «свидетелю». \\
	5.3.8. При напылении проводниковых слоев методом термического испарения время между окончанием напыления подслоя и началом напыления проводникового слоя должно быть не более 2 мин. Напыление медного проводникового слоя на ситалловые и керамические подложки производить с предварительным нагревом подложек до температуры (250 ±20) °С, а на полиимидные подложки — без специального нагрева подложек. Напыление алюминиевого проводникового слоя производить с предварительным нагревом подложек до температуры (120±20)°С. Указанную в КД на плату величину удельного поверхностного сопротивления проводников или их толщину рекомендуется обеспечивать путем подбора массы навески проводникового материала и полным ее испарением. \\
	5.3.9. Ионно-плазменное напыление проводниковых слоев производить с использованием магнетронных систем распыления при давлении аргона в вакуумной камере от 0,13 Па  (1*10-3  мм рт. ст.) до 1,07 Па (8 *10-3 мм рт. ст.). Скорость напыления меди на подложки, предварительно нагретые до температуры (250±20)°С, должна быть не менее 0,8 мкм/мин. Скорость напыления алюминия на подложки, предварительно нагретые до температуры (120 ±20) °С, должна быть не менее 0,6 мкм/мин. Указанную в КД на плату величину удельного поверхностного сопротивления проводников или их толщину рекомендуется обеспечивать временем напыления при поддержании выбранной скорости напыления. \\
	5.3.10. Напыление слоев межуровневой изоляции из неорганического диэлектрика Д014-30 производить ионно-плазменным методом с подачей на мишень высокого напряжения с частотой 13,56 МГц без специального нагрева подложек. Скорость осаждения слоя диэлектрика на подложку должна быть не менее 30 нм/мин. Допускается напыление слоя межуровневой изоляции из неорганического диэлектрика производить через биметаллические маски. Толщину напыляемого слоя диэлектрика рекомендуется обеспечивать временем напыления при поддержании выбранной скорости напыления. \\
	5.3.11. Нанесение слоя двуокиси кремния производить методом плазмохимического осаждения в среде моносилан-кислород со скоростью нанесения не менее 400 нм/ч на подложки, предварительно нагретые до температуры: \\
	1)  (250±10) °С—при изготовлении защитного слоя; \\
	2)  (350±25) °С—при изготовлении слоя межуровневой изоляции. \\
	Толщину напыляемого слоя диэлектрика рекомендуется обеспечивать временем осаждения при поддержании выбранной скорости нанесения. \\
	5.3.12. Напыление защитного слоя из моноокиси германия производить методом термического' испарения на подложки, предварительно нагретые до температуры (120 ±20) °С. \\
	Допускается напыление защитного слоя из моноокиси германия производить через биметаллические маски. Заданную в КД на плату толщину напыляемого слоя диэлектрика рекомендуется контролировать в процессе напыления с помощью приборов контроля толщины. \\
	5.3.13. Нанесение верхнего защитного покрытия проводниковых элементов напылением никеля или ванадия производить в едином- вакуумном  цикле с напылением  проводникового слоя. Режимы напыления верхнего защитного покрытия должны обеспечивать получение установленного КД на плату суммарного удельного поверхностного сопротивления или суммарной толщины для всей проводниковой структуры. \\
	5.3.14. Перед напылением слоев (структур), если напыление следует после фотолитографии, необходимо производить плазмо-химическую обработку подложек в азотной либо в аргоно-кислородной среде при давлении в камере от  133 Па (1 мм рт.ст.) до 266 Па (2 мм рт.ст.). \\
	Обработку подложек, содержащих напыленные резистивные слои или сформированные резисторы, перед напылением проводникового слоя следует производить в аргоновой среде • после предварительном откачки рабочей камеры до остаточного давления не более 1,33*10-3  Па  (1*10-5  мм рт. cт.). \\
	5.3.15. Слои (структуры), полученные вакуумным напылением и плазмохимическим осаждением, должны быть однородного цвета, без царапин, вздутий, отслоений, пор, загрязнений и трещин. Допускаются дефекты, обусловленные дефектами поверхности подложки, разрешенными ТУ на подложки. \\
	5.3.16. Изготовление «свидетелей» на ситалловых подложках производить в следующем порядке: \\
	1) вакуумное напыление проводникового слоя; \\
	2) изготовление фоторезистивной маски; \\
	3) травление пленочной структуры; \\
	4) удаление фоторезистивной маски; \\
	5) разделение подложек на платы. \\
	Допускается изготавливать «свидетели» с использованием биметаллических масок. \\
	5.3.17. Изготовление пленочных структур в едином вакуумном цикле должно включать: \\
	1) очистку подложек (подраздел 5.2); \\
	2) очистку подколпачного устройства установки вакуумного напыления; \\
	3) очистку испарителей; \\
	4) очистку навесок напыляемых материалов; \\
	5) изготовление «свидетелей»; \\
	6) напыление пленочной структуры, установленной КД на плату, без разгерметизации вакуумной  камеры после напыления каждого слоя структуры. \\
	Примечание. Очистку испарителей и навесок производят при использовании метода термического испарения в вакууме. \\
	5.3.18. При изготовлении пленочной структуры в раздельных вакуумных циклах напыление отдельных слоев структуры про- • изводить в обособленных вакуумных циклах одной или разных установок вакуумного напыления. Перед проведением каждого цикла напыления, начиная со второго, необходимо производить межоперационную очистку подложек с напыленным слоем  (слоями)  в соответствии с и. 5.3.14. Допускается межоперационпую очистку не производить, если подложки с напыленным слоем (слоями) не подвергались фотолитографической обработке и хранились не более 6 ч в эксикаторе с силикагелем или не более 10 ч в неразгерметизированной рабочей камере установки вакуумного напыления. При изготовлении резистивно-проводниковой структуры в раздельных вакуумных циклах подложки после напыления резистивного слоя должны быть подвергнуты контролю на соответствие удельного поверхностного сопротивления установленному значению.  \\
	5.4. Требования к изготовлению фоторезистивных масок \\
	5.4.1. Изготовление фоторезистивной маски должно включать и выполняться в указанной последовательности: \\
	1) нанесение фоторезистивного слоя; \\
	2) сушку фоторезистивного слоя; \\
	3) совмещение и экспонирование рисунка; \\
	4) проявление фоторезистивной маски; \\
	5) термообработку фоторезистивной маски. \\
	Для получения двухслойной фоторезистивной маски необходимо изготовление маски в установленном настоящим пунктом порядке произвести дважды. \\
	5.4.2. Для изготовления фоторезистивных масок использовать прецизионные фотошаблоны, соответствующие требованиям ОСТ 4Г 0.073.210-84. \\
	5.4.3. Для изготовления фоторезистивных масок использовать фоторезисты марок: ФП-383. ФП-РН-7, ФП-25, ФП-27-18БС, ФП-051К. \\
	Для изготовления пленочных элементов из диэлектрика Д014-30 рекомендуется использовать фоторезистивную маску из фоторезиста ФП-051К. \\
	Для изготовления отверстий в полиимидных подложках рекомендуется использовать фоторезистивную маску из фоторезиста ФН-11С в смеси с фоторезистом ФН-11С-К \\
	Для ионного травления слоев меди рекомендуется использовать маски: \\
	1) при травлении слоев толщиной до 3 мкм—двухслойную из фоторезиста марки ФП-051К; \\
	2) при травлении слоев толщиной свыше 3 мкм — комбинированную, состоящую из однослойной маски из фоторезиста марки ФП-051К и титановой маски, которая расположена под маской из фоторезиста марки ФП-051К и с ее помощью сформирована. \\
	5.4.4. Изготовление фоторезистивной маски следует выполнять без перерывов. В обоснованных случаях допускаются перерывы между сушкой фоторезистивного слоя и совмещением и экспонированием рисунка, а также между проявлением фоторезистивной маски и термообработкой фоторезистивной маски. При этом длительность каждого перерыва не должна превышать 3 сут, в течение которого подложки должны находиться в светонепроницаемой таре, помещенной в устройство для хранения деталей в защитной среде. \\
	5.4.5. Изготовление фоторезистивных масок на нанесенных на 'подложки слоях без их предварительной химической обработки должно производиться не позднее 3 сут для подложек, находившихся после нанесения слоев в эксикаторе с силикагелем, и не позднее 40 сут для подложек, находившихся после нанесения слоев в  шкафу с защитной средой. \\
	5.4.6. Нанесение фоторезиста на подложки, за исключением полиимидных, производить методом центрифугирования до полного растекания фоторезиста по поверхности подложки. \\
	Толщину нанесенного слоя фоторезиста следует подбирать, исходя из условия устойчивости фоторезистивной маски при травлении пленочных слоев. \\
	Толщины фоторезистивных масок, используемых при ионном травлении, следует подбирать из условия одновременного полного удаления подлежащих травлению участков слоя и возможно более полного удаления самой фоторезистивной маски. \\
	Режимы нанесения в зависимости от вязкости фоторезистов следует подбирать так, чтобы ширина технологического «валика» из фоторезиста по краям подложки была не более 500 мкм.
	5.4.7. Нанесение фоторезиста на полиимидные подложки производить вытягиванием полностью погруженной подложки из ванны с фоторезистом. \\
	5.4.8. Нанесенный слой фоторезиста должен быть ровным, без пузырей, инородных включений и не иметь сквозных дефектов обнажающих подложку или слой, находящийся под фоторезистом!
	5.4.9. Сушку фоторезистивного слоя производить при температуре не более 95 С. Продолжительность сушки не менее: \\
	1) 10 мин — при сушке в термостате; \\
	2) 2 мин - при сушке в инфракрасных лучах. \\
	5.4.10. При выполнении совмещения и экспонирования рисунка, схемы размещения рисунка на подложках, за исключением полиимидных, производить так, чтобы по краям подложки оставалась свободной технологическая зона шириной не менее 500 мкм. \\
	При размещении рисунка схемы на полиимидных подложках должна быть оставлена по периметру подложки технологическая зона, ширина которой обеспечивает размещение в ней базовых' отверстий для совмещения плат между собой в многослойной плате. \\
	При изготовлении маски для травления переходных отверстий экспонирование фоторезиста, нанесенного на обе стороны поли-имидной подложки, производить одновременно для обеих сторон подложки. \\
	5.4.11. После проявления фоторезистивная маска не должна иметь сквозных дефектов, нарушающих ее целостность. Края маски должны быть четкими и ровными, а рисунок полностью соответствовать фотошаблону. На пробельных участках не должно быть остатков недопроявленного фоторезиста. \\
	Допускается ретушь недопроявленных участков и сквозных дефектов маски. \\
	5.4.12. Термообработка фоторезистивной маски, используемой при химическом травлении слоев, должна представлять собой двух-трехступенчатый нагрев подложек с масками. При этом первая ступень должна представлять собой нагрев до температуры (90 ±5) °С с выдержкой не менее 10 мин, а последняя ступень—нагрев до температуры не более 160°С с выдержкой, подбираемой в зависимости от толщины фоторезистивной маски и марки фоторезиста. \\
	5.4.13. Термообработка фоторезистивной маски, используемой при ионном травлении слоев, должна состоять из трех ступеней температуры нагрева: \\
	1) (100±5) °С—выдержка в течение (10±1) мин; \\
	2) (150±5) °С—выдержка в течение (10±1) мин; \\
	3) (180±5) °С—выдержка в течение (30±1) мин. 5.4.14. Удаление фоторезистивной маски после химического травления должно включать: \\
	1) обработку растворами, составы которых определяются маркой удаляемого фоторезиста; \\
	2) плазмохимическую обработку в азотной или аргонокислородной среде при давлении в камере от 1,33*102 Па (1 мм рт. ст.) до 2,66*102 Па (2 мм рт. ст.) без специального подогрева подложек в течение времени, обеспечивающего полное удаление фоторезиста с подложки. Для удаления фоторезистивной маски из смеси фоторезистов ФН-11С и ФН-11С-К  с полиимидных подложек рекомендуется использовать смывающий состав — форсан. Для удаления не прошедшей термообработку некачественной фоторезистивной маски (с целью ее повторения) допускается плазмохимическую обработку не производить. \\
	5.4.15. Для удаления отдельных участков фоторезистивной маски, оставшихся после ионного травления слоев, использовать плазмохимическую обработку. \\
	5.5. Требования к травлению \\
	5.5.1. При изготовлении пленочных элементов схемы травление нанесенных на подложку слоев производить с использованием фоторезистивной маски, изготовленной в соответствии с п. 5.4.1. Качество пленочных элементов, изготовленных травлением (неровность края, сквозные дефекты и др.) определяется качеством фоторезистивной маски и должно соответствовать требованиям, установленным в ТУ или КД на платы. \\
	5.5.2. Размеры пленочных элементов, изготовленных травлением нанесенных на подложку слоев, определяются размерами соответствующих элементов рисунка на фотошаблоне и должны соответствовать КД на плату. \\
	Допускается уход размеров элементов на плате от размеров элементов на фотошаблоне: \\
	1) при химическом травлении не более чем на величину, равную двойной толщине нанесенного слоя (структуры); \\
	2) при ионном травлении не более чем на 2 мкм при толщине слоя меди не более 9 мкм. \\
	5.5.3. Составы для химического травления и режимы травления нанесенных на подложку слоев должны обеспечивать выполнение требований пп. 5.5.1 и 5.5.2. Рекомендуемые составы для неселективного травления слоев из различных материалов приведены в приложении 7. \\
	5.5.4. Ионное травление нанесенных на подложку слоев производить в вакуумной камере после предварительной откачки до остаточного давления не более 1,33*10-3 Па (1*10-5мм рт.ст.) 
	с использованием ионного источника при рабочем давлении аргона не более 5,32*10-2  Па (4*10-4 мм рт. ст.). \\
	Скорость травления медного слоя должна быть не менее: \\
	1) 0,3 мкм/мин—для подложек, неподвижных относительно ионного источника; \\
	2) 0,03 мкм/мин — для подложек, перемещающихся относительно ионного источника. \\
	5.6. Требования к осаждению и анодному окислению слоев \\
	5.6.1. Химическое или электрохимическое осаждение слоя на подложки с напыленными слоями должно включать и выполняться в указанной последовательности с соблюдением требований ГОСТ 9.305-88: \\
	1) обезжиривание; \\
	2) промывку; \\
	3) активацию; \\
	4) промывку; \\
	5) осаждение покрытия; \\
	6) улавливание   драгоценных металлов (для случая их осаждения; \\
	7) промывку; \\
	8) сушку. \\
	5.6.2. Химическое или электрохимическое осаждение слоя на подложки с напыленным слоем меди следует производить сразу после напыления или сразу после удаления с наныленного слоя меди (медных проводниковых элементов схемы) защитного покрытия. \\
	5.6.3. Подготовку поверхности подложек с напыленными слоями (обезжиривание и активацию) производить непосредственно перед помещением подложек в ванну для химического или электрохимического осаждения. \\
	5.6.4. Электрохимическое осаждение слоя меди с целью увеличения толщины напыленного медного слоя и химическое или электрохимическое осаждение защитного покрытия ( никель, олово-висмут, золото и др.) следует, выполнять без перерывов. \\
	В обоснованных-случаях допускается перерыв не более 12 ч, в течение которого платы с осажденным слоем меди должны находиться в эксикаторе с силикагелем. \\
	5.6.5. Анализ электролитов, используемых для осаждения слоев, производить по ОСТ 107.460092.001-86. \\
	5.6.6. Промывку подложек с напыленными слоями после активации производить в проточной дистиллированной или дспонизован-ной воде. Продолжительность промывки — от 3 до 5 с.
	5.6.7. Слои, полученные химическим или электрохимическим осаждением, должны быть однородного цвета, без отслоений, пузырей, трещин и загрязнений.
	Увеличение геометрических размеров пленочных элементов после электрохимического осаждения не должно превышать 30\% от толщины осажденного слоя. \\
	5.6.8. Анодное окисление напыленных слоев алюминия производить по ГОСТ 9.305-88. \\
	5.7. Требования к подгонке резисторов \\
	5.7.1. Подгонку сопротивлений резисторов до значений, установленных КД на плату, производить токовым, электроэрозионным, ультразвуковым, лазерным методами и методом электрохимического анодного окисления. Токовый метод применяют для подгонки резисторов, полученных термическим испарением в вакууме, при отсутствии жестких требований к постоянству отношений значений сопротивлений резисторов на плате. Метод электрохимического анодного окисления применяют для танталовых резисторов. Состав для электрохимического анодного окисления тантала и способ его приготовления приведены в приложении 8. \\
	5.7.2. Подгонка сопротивления не должна приводить к уменьшению площади резистора, превышающему норму, установленную в КД на плату, или к появлению локальных зон, в которых рассеиваемая удельная мощность превосходит допустимую по ТУ на материал резистора, а также не должна приводить к появлению дефектов внешнего вида резисторов, не отвечающих требованиям ТУ или КД на платы. \\
	
	
	\begin{thebibliography}{}
		\bibitem{1} Твердотельные устройства в технике связи/ Л.Г. Гассанов и др. – М.: Радио и связь, 1988.
		\bibitem{2} Малорацкий Л.Г., Микроминиатюризация элементов и устройств СВЧ., М., «Сов. Радио», 1976.
		\bibitem{3} Бушминский И.П., Гудков А.Г., Дергачев В.Ф. Конструкторское проектирование микросхем СВЧ: Учеб. пособие. М.: Изд-во МГТУ им. Н.Э.Баумана, 1991, 225 с.
		\bibitem{4} Справочник по расчету и конструированию СВЧ полосковых устройств / С.И.Бахарев, В.И.Вольман и др.: Под. ред. В.И.Вольмана. М.: Радио и связь, 1982, 328 с.
		\bibitem{5} Полупроводниковые приборы. Сверхвысокочастотные диоды. Справочник /Б.А. Наливайко и др. Под ред. Б.А. Наливайко. – Томск: МГП «РАСКО», 1992.
		\bibitem{6} ГОСТ 2.734 – 68. Обозначения условные графические в схемах. Линии сверхвысокой частоты и их элементы.
		\bibitem{7} ОСТ 107.750 878.002 – 87   Технология изготовления толстопленочных плат.
		\bibitem{8} \href{http://mart7157.narod.ru/voise_10.html}{
			Изображение~c~рисунка~\ref{img:The_principle_of_operation_of_the_amplitude_detector}
		}
		\bibitem{9} 3A206A-6 параметры диода
	\end{thebibliography}
	
	
\end{document}





























