\documentclass{bmstu}
\usepackage{multirow}
%\usepackage{karnaugh-map}
\usepackage{tikz}
\usepackage{float}

\usetikzlibrary{karnaugh}

\begin{document}

	\makecourseworktitle
	{Радиоэлектроника и лазерная техника (РЛ)} % Название факультета
	{Технология приборостроения (РЛ6)} % Название кафедры
	{Проектирование конструкции микрополоскового детектора} % Тема работы
	{Филимонов~С.~В./РЛ6-61} % Номер группы/ФИО студента (если авторов несколько, их необходимо разделить запятой)
	{Федоркова Н.В.} % ФИО научного руководителя
	{}
	
	\tableofcontents

	\chapter{Теория}
	
	\chapter{Принцип работы устройства.}
	
	\chapter{Синтез топологии платы в программе AWR Design Environment 2009.}
	
	\chapter{Расчет рабочих параметров.}
	
	\chapter{Анализ влияния ПД на параметры диода на рабочие характеристики устройства.}
	
	\chapter{Технология изготовления.} 
	
	
	\begin{thebibliography}{}
		\bibitem{litlink1} Твердотельные устройства в технике связи/ Л.Г. Гассанов и др. – М.: Радио и связь, 1988.
		\bibitem{litlink2} Малорацкий Л.Г., Микроминиатюризация элементов и устройств СВЧ., М., «Сов. Радио», 1976.
		\bibitem{litlink3} Бушминский И.П., Гудков А.Г., Дергачев В.Ф. Конструкторское проектирование микросхем СВЧ: Учеб. пособие. М.: Изд-во МГТУ им. Н.Э.Баумана, 1991, 225 с.
		\bibitem{litlink4} Справочник по расчету и конструированию СВЧ полосковых устройств / С.И.Бахарев, В.И.Вольман и др.: Под. ред. В.И.Вольмана. М.: Радио и связь, 1982, 328 с.
		\bibitem{litlink5} Полупроводниковые приборы. Сверхвысокочастотные диоды. Справочник /Б.А. Наливайко и др. Под ред. Б.А. Наливайко. – Томск: МГП «РАСКО», 1992.
		\bibitem{litlink6} ГОСТ 2.734 – 68. Обозначения условные графические в схемах. Линии сверхвысокой частоты и их элементы.
		\bibitem{litlink7} ОСТ 107.750 878.002 – 87   Технология изготовления толстопленочных плат.
	\end{thebibliography}
	
%	На~рисунке~\ref{img:BCD} пример семисегментного индикатора.
%	\includeimage
%	{BCD} % Имя файла без расширения (файл должен быть расположен в директории inc/img/)
%	{f} % Обтекание (без обтекания)
%	{h} % Положение рисунка (см. figure из пакета float)
%	{0.25\textwidth} % Ширина рисунка
%	{Семисегментный индикатор} % Подпись рисунка 	
	
\end{document}





























