\documentclass{bmstu}
\usepackage{multirow}
%\usepackage{karnaugh-map}
\usepackage{tikz}
\usepackage{float}

\usetikzlibrary{karnaugh}

\begin{document}
	\makereporttitle
	{Радиоэлектроника и лазерная техника (РЛ)} % Название факультета
	{Технология приборостроения (РЛ6)} % Название кафедры
	{домашней работе №1} % Название работы (в дат. падеже)
	{Цифровые устройства и микропроцессоры} % Название курса (необязательный аргумент)
	{ПЛИС Altera ССИ, минимизация алгебраических функций} % Тема работы
	{20Л274} % Номер варианта (необязательный аргумент)
	{Филимонов~С.~В./РЛ6-61} % Номер группы/ФИО студента (если авторов несколько, их необходимо разделить запятой)
	{Семеренко~Д.~А.} % ФИО преподавателя
	
		\tableofcontents
	%Реализовать шифратор для вывода знака на ССИ:
	%	1.1 написать алгебраические уравнения в СКНФ и СДНФ;
	%	1.2 минимизировать с помощью: законов алгебры логики, карт Карно, метода Квайна;
	%	1.3 привести полученныя выражения к базису 2И-НЕ/2ИЛИ-НЕ;
	%	1.4 начертить цифровую схему.
	\chapter{Реализация шифратора для вывода знака на ССИ.}
	На~рисунке~\ref{img:BCD} пример семисегментного индикатора.
	
	\includeimage
	{BCD} % Имя файла без расширения (файл должен быть расположен в директории inc/img/)
	{f} % Обтекание (без обтекания)
	{h} % Положение рисунка (см. figure из пакета float)
	{0.25\textwidth} % Ширина рисунка
	{Семисегментный индикатор} % Подпись рисунка 
	
	Кодировка	
	
\end{document}





























