\documentclass{bmstu}
\usepackage{multirow}
%\usepackage{karnaugh-map}
\usepackage{tikz}
\usepackage{float}

\usetikzlibrary{karnaugh}

\begin{document}

	\makecourseworktitle
	{Радиоэлектроника и лазерная техника (РЛ)} % Название факультета
	{Технологии приборостроения (РЛ6)} % Название кафедры
	{Проектирование конструкции микрополоскового детектора} % Тема работы
	{Филимонов~С.~В./РЛ6-61} % Номер группы/ФИО студента (если авторов несколько, их необходимо разделить запятой)
	{Федоркова Н.В.} % ФИО научного руководителя
	{}
	
	\tableofcontents
	
	\chapter{Условие.}
	
	\begin{center}
	\textbf{Московский государственный технический университет им.Н.Э.Баумана}
	
	\textbf{Кафедра «Технология приборостроения»}
	
	\textbf{Техническое задание} \\
	на курсовую работу по курсу «Устройства СВЧ и антенны»
	\end{center}
	
\noindent \textbf{Ф.И.О. студента:}  Филимонов С.В.
\noindent \textbf{Группа:}     РЛ6-61
	
\noindent \textbf{Тема работы:}  Проектирование конструкции микрополоскового детектора
	
\noindent \textbf{Задание по конструкторской части} \\
\indent Разработать конструкцию микрополоскового детектора мощности на базе копланарных линий передачи и диода 3А206 А-6.
	
\noindent \textbf{Исходные данные:}\\
\indent Требования к электрическим параметрам: частота сигнала 8 +/- 0,3 ГГц. Материал подложки – поликор. Технология изготовления – толстопленочная. Ориентировочные габариты: 10 х 15 х 0,5 мм.
	
\noindent \textbf{Объем работы:} \\
\indent графической части -  3 листа,\\
\indent расчетно-пояснительной записки – 35 - 50 листов.
	
\noindent \textbf{Содержание графической части:}
	
\indent Лист 1. Эскиз микрополосковой платы.\\
\indent Лист 2. Электрическая принципиальная схема.\\ 
\indent Лист 3. Эскизы конструкций ЭРЭ.
	
\noindent \textbf{Содержание расчетно-пояснительной записки}
	
\noindent1. Принцип работы устройства. \\
\noindent2. Синтез топологии платы в программе MWO.  \\
\noindent3. Расчет рабочих параметров. \\
\noindent4. Анализ влияния ПД на параметры диода на рабочие характеристики устройства.\\
\noindent5. Технология изготовления.\\
	
\noindent \textbf{Руководитель работы:} Федоркова Н.В.
	
\noindent \textbf{Студент:} Филимонов С.В.
	
\noindent \textbf{Дата выдачи задания:} 24.02.2023
	
	
	\chapter{Принцип работы устройства.}
	
	Амплитудный детектор – это устройство, предназначенное для получения на
	выходе напряжения, изменяющегося в соответствии с законом модуляции
	амплитуды входного гармонического сигнала. Процесс детектирования
	амплитудно-модулированных (АМ) сигналов вида
	
	\begin{center}
		 $U_c(t) = u_a(t) \cdot cos(\omega _c \cdot t), \text{где } u_a(t)=U_c[1 + max(t)],$
	\end{center}
	
	где $m_a$ < 1 - коэффициент глубины модуляции, $U_c$ амплитуда несущего 
	колебания с частотой $\omega _c$, заключается в воспроизведении модулирующего 
	сообщения $x(t)$ с наименьшими искажениями. Спектр сообщения $x(t)$ 
	сосредоточен в области низких частот (частот модуляции), а спектр сигнала 
	$U_c(t)$ – в области частоты $\omega _c$, значение которой обычно намного 
	превышает значение наивысшей частоты модуляции. Преобразование спектра при 
	демодуляции возможно только в устройствах, выполняющих нелинейное или 
	параметрическое преобразование входного сигнала $U_c(t)$. В качестве подобного 
	устройства может быть использован диод или транзистор.
	
	\includeimage
		{The_principle_of_operation_of_the_amplitude_detector}
		{f} % Обтекание (без обтекания)
		{h} % Положение рисунка (см. figure из пакета float)
		{0.75\textwidth} % Ширина рисунка
		{Принцип работы детектора.} % Подпись рисунка
	
	При использовании нелинейного устройства, обладающего квадратичной 
	вольт‐амперной характеристикой, выходной ток имеет вид
	\begin{center}
		$i = B u_c^2(t) = Bu_a^2(t)[0.5 + 0.5 cos(2 \omega _c t)],$
	\end{center}
	
	где $В$ - постоянный коэффициент. После устранения фильтром низких частот
	(ФНЧ) составляющей с частотой $2\omega_c$ получим
	\begin{center}
		$i = 0.5 B U_c^2[1 + 2m_a x(t) + m_a^2 x^2(t)].$
	\end{center}
\indent	В этом токе содержится составляющая вида $BU_c^2m_ax(t)$, пропорциональная
	передаваемому сообщению, а также составляющая $0.5BU_c^2m_a^2x^2(t)$
	, которая определяет степень нелинейных искажений модулирующего сообщения
	$x(t)$. \\
	Параметрическое преобразование осуществляется путем умножения $U_c(t)$ на
	опорное колебание, имеющее вид 
	\begin{center}
		$U_0(t) = U_0 cos(\omega_c t).$
	\end{center}
	В этом случае результат перемножения определяется следующим выражением
	\begin{center}
		$U_c(t)U_0(t) = u_a(t)U_0[0.5 + 0.5cos(2\omega_c t)].$
	\end{center}
\indent	Составляющая с частотой $2\omega_c$ устраняется ФНЧ и в результате 
	формируется низкочастотный сигнал вида $0.5 U_0 u_a(t)$. Отделяя постоянную 
	составляющую $0.5 U_0 U_c$, например, при помощи разделительного конденсатора, 
	получаем сигнал вида $0.5 U_0 U_c max(t)$, форма которого определяется 
	передаваемым сообщением $x(t)$. \\
\indent	Амплитудный детектор, выполняемый по микрополосковой технологии, состоит из
	элемента связи с СВЧ‐трактом (согласующего устройства), диода, ФНЧ и вывода
	сигнала на НЧ. \\
\indent	Значение тока через диод $i_g$ для режима покоя $(u_c(t)=0)$ может быть 
	найдено из уравнений
	\begin{equation*}
		\begin{cases}
			 i_g = f(U_g)\\
			 i_g = - \frac{u_\text{вых}}{R_\text{н}}
		\end{cases}
	\end{equation*}
\indent	где $U_g$ – напряжение на диоде. Первое уравнение является уравнением 		
	вольтамперной характеристики (ВАХ) диода как безынерционного нелинейного 
	элемента. Из‐за нелинейного характера ВАХ, форма тока через диод $i_g$ при 
	синусоидальной форме сигнала $U_c(t)$ не является синусоидальной. В составе 
	тока появляется постоянная составляющая, которая, протекая по резистору 
	$R_\text{н}$, создает падение напряжения $U_g$, смещающая положение рабочей 
	точки. При увеличении амплитуды входного напряжения смещение рабочей точки 
	увеличивается, и ток через диод будет приближаться по форме к однополярным 
	импульсам, открывающим диод при положительных значениях входного напряжения.
	 
	\includeimage
		{Graph_teory}
		{f} % Обтекание (без обтекания)
		{h} % Положение рисунка (см. figure из пакета float)
		{0.75\textwidth} % Ширина рисунка
		{Детектирование АМ сигналов.} % Подпись рисунка
	
\indent	На~рисунке~\ref{img:Graph_teory} приведены 
	формы
	напряжений и токов на входе детектора для двух случаев, когда амплитуды входных
	сигналов удовлетворяют неравенству $U_c(1) < U_c(2)$. Тогда постоянные 
	составляющие напряжений $U_c(1) < U_c(2)$ и $I_c(1) < I_c(2)$. На этом же 
	рисунке изображена зависимость $i_g = f(t)$. \\
\indent	Вольтамперная характеристика диода в широком диапазоне токов достаточно 
	точно аппроксимируется экспоненциальной зависимостью
	\begin{center}
		$i_g = I_\text{об} (e^{u_g(t)/\phi_T} - 1),$
	\end{center}

\indent где $I_\text{об}$ – абсолютное значение величины обратного тока диода,
	$\phi_T$ – температурный потенциал, равный при $T = 293^{\circ}$ примерно 26
	мВ. Из этой зависимости следует, что \\
\indent	‐ с увеличением $R_\text{н}$ увеличивается крутизна детекторной
	характеристики, \\
\indent	‐ с увеличением уровня сигнала снижается нелинейность детекторной 
	характеристики. \\
\indent	Из этого следует, что диодный детектор работает в двух режимах когда на вход
	поступает «слабый» сигнал и когда – «сильный». В режиме «слабого» сигнала
	характеристика диода аппроксимируется квадратичной зависимостью, в режими 
	сильных токов – линейной зависимостью.
	
	
	\chapter{Синтез топологии платы в программе AWR Design Environment.}
	
	Амплитудный детектор, выполняемый по микрополосковой технологии, состоит из 
	копланарной линии, диода, фильтра и вывода сигнала на НЧ. Т.к. информационный 
	сигнал имеет частоту в полосе от 7700 МГц до 8300МГц, как следует из условия, 
	нужно использовать полосно‐пропускной фильтр. \\
\indent	Параметры диода и подложки выбираются по техническому заданию. Подложка – 
	поликор $(e_r = 9.8; T_{and} = 0.0003)$. Диод – 3А206А-6 \cite{8}.
	
\includeimage
	{diod_3a206_6}
	{f} % Обтекание (без обтекания)
	{H} % Положение рисунка (см. figure из пакета float)
	{1.0\textwidth} % Ширина рисунка
	{Параметры диода 3A206A-6.} % Подпись рисунка
	
\indent	Рассчитаем детктор в программе AWR Design.

%\includeimage
%	{Circuit_on_the_coplanar_line_new}
%	{f} % Обтекание (без обтекания)
%	{H} % Положение рисунка (см. figure из пакета float)
%	{1.0\textwidth} % Ширина рисунка
%	{Схема микрополоскового детектора на копланарной линии.} % Подпись рисунка

	Для расчёта копланарной линии с R=50 Ом прибегнем к помощи инструмента TXLine.
	
\includeimage
	{TXLine_coplanar_line}
	{f} % Обтекание (без обтекания)
	{H} % Положение рисунка (см. figure из пакета float)
	{1.0\textwidth} % Ширина рисунка
	{Результат расчёта копланарной линии.} % Подпись рисунка

	Заметим, что результаты вычисления копланарной линии не удовлетворяют нас по
	предельным возможностям технологий. Введём изменение – уменьшим толщину
	проводящего слоя до 10 мкм.

\includeimage
	{TXLine_coplanar_line_new}
	{f} % Обтекание (без обтекания)
	{h} % Положение рисунка (см. figure из пакета float)
	{1.0\textwidth} % Ширина рисунка
	{Откорректированные результаты расчёта.} % Подпись рисунка
	
	Введём полученные значения в схему.
	
\includeimage
	{Circuit_on_the_coplanar_line_new}
	{f} % Обтекание (без обтекания)
	{H} % Положение рисунка (см. figure из пакета float)
	{1.0\textwidth} % Ширина рисунка
	{Схема с обновленными параметрами.} % Подпись рисунка

	Построим график передачи мощности.
	
\includeimage
	{Graph_LSSnm_measurement}
	{f} % Обтекание (без обтекания)
	{H} % Положение рисунка (см. figure из пакета float)
	{0.90\textwidth} % Ширина рисунка
	{Параметры графика.} % Подпись рисунка

	В результате получим:
% график LSSnm
\includeimage
	{Graph_LSSnm}
	{f} % Обтекание (без обтекания)
	{H} % Положение рисунка (см. figure из пакета float)
	{1.0\textwidth} % Ширина рисунка
	{График LSSnm.} % Подпись рисунка
	
	Из графика не понятно где происходит переход, зададим границы от -10 до -3 дБ.
	В результате получим:
	
\includeimage
	{Graph_LSSnm_new}
	{f} % Обтекание (без обтекания)
	{H} % Положение рисунка (см. figure из пакета float)
	{0.8\textwidth} % Ширина рисунка
	{График LSSnm.} % Подпись рисунка
	
	Исследуя зависимость от тока, согласно полученным данным характеристика имеет линейный характер на участке от -10 дБ до -3 дБ.
	
	\section{Граффик тока.}

	% график LSSnm
	\includeimage
	{Icurrent_table}
	{f} % Обтекание (без обтекания)
	{H} % Положение рисунка (см. figure из пакета float)
	{1.0\textwidth} % Ширина рисунка
	{Настройки для графика тока.} % Подпись рисунка
	
	\includeimage
	{Icurrent_plot}
	{f} % Обтекание (без обтекания)
	{H} % Положение рисунка (см. figure из пакета float)
	{0.8\textwidth} % Ширина рисунка
	{График Icurrent.} % Подпись рисунка

	\chapter{Расчет рабочих параметров.}
	
	\indent	Параметры диода и подложки выбираются по техническому заданию. Подложка – 
	поликор $(e_r = 9.8; T_{and} = 0.001)$. Диод – 3А206А-6 \cite{8}.
	
	\includeimage
	{diod_3a206_6}
	{f} % Обтекание (без обтекания)
	{H} % Положение рисунка (см. figure из пакета float)
	{1.0\textwidth} % Ширина рисунка
	{Параметры диода 3A206A-6.} % Подпись рисунка
	
	\includeimage
	{detector_circuit}
	{f} % Обтекание (без обтекания)
	{H} % Положение рисунка (см. figure из пакета float)
	{1.0\textwidth} % Ширина рисунка
	{Принципиальная схема микрополосового детектора.} % Подпись рисунка
	
	\includeimage
	{detector_IC}
	{f} % Обтекание (без обтекания)
	{H} % Положение рисунка (см. figure из пакета float)
	{1.0\textwidth} % Ширина рисунка
	{Чертеж микрополосового детектора.} % Подпись рисунка
	
	\chapter{Анализ влияния ПД на параметры диода и рабочие характеристики устройства.}
	
	Проводить анализ влияние ПД на параметры диода и на рабочие характеристики устройства не нужно.
	
	\chapter{Технология изготовления.} 
	
	ОСТ 107.750878.002-87   «Технология изготовления толстопленочных плат» приводит следующую схему (порядок выполнения операций)  изготовления микрополосковой платы, содержащей резистивные элементы и тонкоплёночные конденсаторы:\\
	
	\noindent 1. Отмывка подложек (подраздел 5.1.6) \\
	2. Изготовление проводников, нижних обкладок конденсаторов (подраздел 5.1.3) \\
	3. Изготовление диэлектрика конденсаторов (подраздел 5.1.3) \\
	4. Изготовление межуровневой изоляции (подраздел 5.1.3) \\
	5. Изготовление проводников, верхних обкладок конденсаторов и контактных площадок (подраздел 5.1.3) \\
	6. Изготовление защитного слоя (подраздел 5.1.3) \\
	7. Изготовление резисторов (подраздел 5.1.4) \\
	8. Контроль плат (ОСТ 107.750878.002-87) \\

\iffalse	% старое
\noindent 	1. Отмывке подложек (подраздел 5.2) \\ 
	2. Вакуумное напыление резистивного слоя (подраздел 5.3) \\
	3. Контроль качества резистивного слоя (пп. 6.6 и 6.7 ) \\
	4. Изготовление фоторезистивной маски (подраздел 5.4) \\
	5. Травление резистивного слоя (подраздел 5.5) \\
	6. Удаление фоторезистивной маски (подраздел 5.4) \\
	7. Вакуумное напыление структуры ванадий-алюминий  (подраздел 5.3) \\
	8. Изготовление фоторезистивной маски (подраздел 5.4) \\
	9. Травление проводниковой структуры (подраздел 5.5) \\
	10. Удаление фоторезистивной маски (подраздел 5.4) \\
	11. Нанесение диэлектрического слоя (подраздел 5.3) \\
	12. Изготовление фоторезистивной маски (подраздел 5.4) \\
	13. Травление диэлектрического слоя (подраздел 5.5) \\
	14. Удаление фоторезистивной маски (подраздел 5.4) \\
	15. Вакуумное напыление слоя алюминия  (подраздел 5.3) \\
	16. Изготовление фоторезистивной маски (подраздел 5.4) \\
	17. Травление слоя алюминия (подраздел 5.5) \\
	18. Удаление фоторезистивной маски (подраздел 5.4) \\
	19. Вакуумное напыление проводниковой структуры (подраздел 5.3) \\
	20. Изготовление фоторезистивной маски (подраздел 5.4) \\
	21. Травление проводниковой структуры (подраздел 5.5) \\
	22. Удаление фоторезистивной маски (подраздел 5.4) \\
	23. Контроль плат (ОСТ 107.750871.001-86) \\
	24. Подгонка резисторов (подраздел 5.7) \\
	25. Изготовление защитного слоя (подраздел 5.4) \\
	26. Химическое осаждение олова (подраздел 5.6) \\
	27. Разделение подложек на платы (подраздел 5.8) \\
	28. Контроль плат (ОСТ 107.750871.001-86) \\
\fi

	\indent Рассмотрим два основных метода нанесения, применяемых для рассматриваемого устройства: вакуумное напыление и гальваническое наращивание.\\
	\indent Вакуумная технология обеспечивает получение пленок с заданными электрофизическими свойствами и хорошей адгезией на полированных диэлектрических подложках. Сущность метода термического испарения в вакууме состоит в том, что при температуре, когда давление собственных паров испаряемого вещества значительно превышает силу сцепления между атомами, происходит термическое испарение материала. При этом в сторону подложки направляется прямолинейный молекулярный поток испаряемого вещества.
	
	\includeimage
	{vacum}
	{f} % Обтекание (без обтекания)
	{H} % Положение рисунка (см. figure из пакета float)
	{0.65\textwidth} % Ширина рисунка
	{Схема подколпачного устройства установки термического испарения в вакууме.} % Подпись рисунка
	
	\indent Согласно схеме \ref{img:vacum}, из испарителя 1 испаряемое вещество 2 осаждается на подложку 3 или на заслонку 4, контролирующую начало и окончание осаждения материала на подложку. Подложка подогревается с помощью нагревателя 5. Во время испарения контролируется температура подложки, температура испарителя, скорость конденсации испаряемого вещества, толщина пленок, давление остаточных газов и т.д., для чего вакуумные установки оснащены специальными датчиками и приборами. За один цикл откачки возможно нанесение нескольких материалов при смене испарителей. Возможно внедрение т.н. «каруселей» для одновременной обработки нескольких подложек.\\
	\indent Гальваническое наращивание основано на электролизе растворов пол действием электрического тока и осаждении металла на катоде. Методом электролитического осаждения изготовляют токопроводящие элементы схемы (Cu, Ag) и защитные антикоррозионные покрытия (Ni, Au, Sn-Bi, Sn-Co и др.).\\
	\indent В зависимости от технологии осаждение слоя металла проводится по всей поверхности подложки (субтрактивная технология) или по сформированному рисунку схемы, соединенному в единую электрическую цепь с помощью технологических перемычек. Равномерность нанесения электролитических покрытий зависит от геометрических и электрохимических условий из осаждения. При заниженных плотностях тока возможно утоньшение покрытия в середине платы, при завышенных – образование утолщений и шероховатостей на углах и торцах платы. Для увеличения производительности, а также получения более плотных и качественных покрытий применяют реверсирование тока, т.е. периодическое переключение полюсов на шинах ванн с помощью автоматического реле времени. 
	
	\includeimage
	{3a206}
	{f} % Обтекание (без обтекания)
	{H} % Положение рисунка (см. figure из пакета float)
	{0.65\textwidth} % Ширина рисунка
	{Схема диода 3A206A-6.} % Подпись рисунка
	
	\begin{thebibliography}{}
		\bibitem{1} Твердотельные устройства в технике связи/ Л.Г. Гассанов и др. – М.: Радио и связь, 1988.
		\bibitem{2} Малорацкий Л.Г., Микроминиатюризация элементов и устройств СВЧ., М., «Сов. Радио», 1976.
		\bibitem{3} Бушминский И.П., Гудков А.Г., Дергачев В.Ф. Конструкторское проектирование микросхем СВЧ: Учеб. пособие. М.: Изд-во МГТУ им. Н.Э.Баумана, 1991, 225 с.
		\bibitem{4} Справочник по расчету и конструированию СВЧ полосковых устройств / С.И.Бахарев, В.И.Вольман и др.: Под. ред. В.И.Вольмана. М.: Радио и связь, 1982, 328 с.
		\bibitem{5} Полупроводниковые приборы. Сверхвысокочастотные диоды. Справочник /Б.А. Наливайко и др. Под ред. Б.А. Наливайко. – Томск: МГП «РАСКО», 1992.
		\bibitem{6} ГОСТ 2.734 – 68. Обозначения условные графические в схемах. Линии сверхвысокой частоты и их элементы.
		\bibitem{7} ОСТ 107.750 878.002 – 87 - Технология изготовления толстопленочных плат.
		%\bibitem{8} \href{http://mart7157.narod.ru/voise_10.html}{Изображение~c~рисунка~\ref{img:The_principle_of_operation_of_the_amplitude_detector}}
		\bibitem{8} 3A206A-6 параметры диода
	\end{thebibliography}
	
	
\end{document}





























